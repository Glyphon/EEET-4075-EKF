\documentclass{ieeeaccess}
\usepackage{cite}
\usepackage{amsmath,amssymb,amsfonts}
\usepackage{algorithmic}
\usepackage{graphicx}
\usepackage{units}
\usepackage{color}
\usepackage{textcomp}
\usepackage{caption}
\usepackage{etoolbox}
\usepackage{placeins}

\begin{document}

\history{Date of current version Monday 20, 2020.}
\doi{PLACEHOLDER}

\title{EEET 4075 Mechatronic System Design 2 - Project Report}
\author{\uppercase{Michael J. Duke}\authorrefmark{1}}
\address[1]{University of South Australia, Mawson Lakes, SA 5095 Australia (e-mail: dukmj002@mymail.unisa.edu.au)}
\tfootnote{This work was supported in part by the University of South Australia}

\markboth
{Michael J. Duke: EEET 4075 Mechatronic System Design 2 - Project Report}
{Michael J. Duke: EEET 4075 Mechatronic System Design 2 - Project Report}

\titlepgskip=-15pt

\maketitle

\section{Introduction}
\label{sec:introduction}
\PARstart{P}{ath} planning, open loop, closed loop controllers advantages and disadvantages.\par

\section{Methodology}
\label{sec:meth}
Open loop controllers advantages and disadvantages\par
Closed loop controllers advantages adn disadvantages\par
The steps required to navigate are create an obstacle map, create a valid path map, path through map.\par
FindFinalPath - This function loads the file 'map' which contains the obstacles to avoid, the map is inflated by the radius of the robot so that the geometry of the robot does not need to be considered when determining collision, and it can be modeled simply as a point representing its centre. Then using the matrix of waypoint coordinates it determines the path from each waypoint to the next using the FindLocalPath function explained later and concatinates that to a single matrix of coordinates representing the final path.
FindLocalPath - This function begins by creating a probibilistic roadmap (PRM) with the inflated map created in the FindFinalPath function. It then attempts to find a path between two coordinates, also passed from FindFinalPath, and if one cannot be found, it increases the number of nodes, regenerates the roadmap, and tries again. This process repeats until a feasible path can be found, this path is then returned to the FindFinalPath function to be concatinated.

\section{Results and discussion}
\label{sec:res}

PRM is not a path planning algorithm in and of itself, it simply creates a map of interconnected nodes that can then be used by a path planning algorithm to find a path. The actual path planning algorithm used by the findpath() function is an A* algorithm, which, put simply, tests what would be the shortest possible path from start to finish, a straight line, until an obstacle or the finish is encountered. If the finish has not been encountered it then tests the paths using nodes on either side of the previous path until, once again, an obstacle or the finish is encountered. This process repeats until the finish is actually encountered.\par
\section{Conclusion}
\label{sec:con}

\EOD

\end{document}
