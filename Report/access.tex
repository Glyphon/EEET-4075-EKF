\documentclass{ieeeaccess}
\usepackage{cite}
\usepackage{amsmath,amssymb,amsfonts}
\usepackage{algorithmic}
\usepackage{graphicx}
\usepackage{units}
\usepackage{color}
\usepackage{textcomp}
\usepackage{caption}
\usepackage{etoolbox}
\usepackage{placeins}

\begin{document}

\history{Date of current version Monday 20, 2020.}
\doi{PLACEHOLDER}

\title{EEET 4075 Mechatronic System Design 2 - Project Report}
\author{\uppercase{Michael J. Duke}\authorrefmark{1}}
\address[1]{University of South Australia, Mawson Lakes, SA 5095 Australia (e-mail: dukmj002@mymail.unisa.edu.au)}
\tfootnote{This work was supported in part by the University of South Australia}

\markboth
{Michael J. Duke: EEET 4075 Mechatronic System Design 2 - Project Report}
{Michael J. Duke: EEET 4075 Mechatronic System Design 2 - Project Report}

\titlepgskip=-15pt

\maketitle

\section{Introduction}
\label{sec:introduction}
\PARstart{M}{ultirate} Extended Kalman Filters and absolute versus relative (dead reckoning) localisation.

\section{Related Works}
\label{sec:rel}
Several different ways to combine the estimations of different sensors, difficulties with multi-rate systems. Variants or modifications on EKF and UKF, compare and contrast, computational cost, relative accuracy, limitations. maybe compare to particle filters as well? Need for an initial estimate of position and heading. Beacon based navigation.

\section{Methodology}
\label{sec:meth}
	
\subsection{Odometry}

	\begin{equation}
	\label{eq:xvec1}
		\dot{\boldsymbol{x}} = 
		\begin{bmatrix}
			\dot{x}		\\
			\dot{y}		\\
			\dot{\theta}
		\end{bmatrix}
		=
		\begin{bmatrix}
			\frac{r}{2}\left(\omega_{l} + \omega_{r}\right)\cos{\theta}	\\
			\frac{r}{2}\left(\omega_{l} + \omega_{r}\right)\sin{\theta}	\\
			\frac{r}{b}\left(\omega_{r} - \omega_{l}\right)
		\end{bmatrix}
	\end{equation}
\section{Results and Discussion}
\label{sec:res}


\section{Conclusion}
\label{sec:con}


\section{References}

\EOD

\end{document}
