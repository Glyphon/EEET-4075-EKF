In simulation, the method used above showed very promising results, with its biggest downfalls showing the possibility of being rectified. The systemic error in the state estimates was not due to the filtering process itself, and such should be fixable with a more accurate model of how the LiDAR works both in simulation and physically. The other issue was that of the update rate of the filter, due to the experiment being run entirely within MATLAB with local parallel processes, as well as the simulation running in a virtual machine running on the same machine, the fastest update rate achievable was 5Hz for the primary measurements, and 1Hz for the secondary. This was not a limitation of the dependent topics, as they updated at a rate of at least 25Hz based off initial tests. If the code was written in something like Python, and the physical TurtleBot used instead of the simulation, the processing could be more easily paralleled and distributed to offload the computation.\par
The methods as they stand are suitable as are as part of a more complex path planning and motion control algorithms as long as the systemic errors mentioned are taken into account for tolerances.